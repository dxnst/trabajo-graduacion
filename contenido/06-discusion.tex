\chapter{Discusión y análisis}
Este estudio tuvo como objetivo establecer la asociación entre factores 
sociodemográficos, económicos, familiares y médicos con el riesgo en el 
neurodesarrollo en niños menores de 5 años que asisten a servicios de atención 
primaria en el distrito de Quetzaltenango, mediante evaluaciones con el
Cuestionario Edades y Etapas 3. Los resultados obtenidos fueron sometidos a
pruebas de normalidad de Kolmogorov-Smirnov, confirmando distribuciones
aproximadamente normales para todos los dominios del desarrollo 
(Cuadro~\ref{tab:estadisticas_z_scores}). Posteriormente se realizó un análisis 
descriptivo y un análisis inferencial utilizando chi-cuadrado y ANOVA para
identificar grupos de riesgo y su significancia estadística.

La muestra de 1,725 niños menores de 5 años mostró una distribución por sexo 
con predominancia masculina (53.45\%) sobre femenina (46.55\%), como se observa
en el Cuadro~\ref{tab:resultado1}. La distribución etaria reveló mayor
concentración en los grupos de 7-12 meses (19.7\%), lo que refleja las
características de la población que acude a servicios de atención primaria para
controles de crecimiento y vacunación. 

En cuanto a la edad gestacional al nacer, el 89.5\% fueron nacimientos a
término, con un 10.5\% de prematuridad en diferentes grados
(Cuadro~\ref{tab:eg}). Similar a estudios en Guatemala que estiman 12.6\% de
partos pretérmino \cite{Pusdekar2020}.

La distribución étnica mostró 57.04\% de población no indígena y 42.96\%
indígena, con marcada concentración de población indígena en área rural
(71.2\%), mientras que el 85.1\% de la muestra total reside en área urbana 
(Cuadro~\ref{tab:etnia-residencia}). Esta distribución refleja las
características demográficas del departamento de Quetzaltenango y es fundamental
para interpretar los resultados del neurodesarrollo considerando diferencias
demográficas y socioeconómicas.

Las características familiares revelaron que las edades parentales se
concentraron entre 25-29 años (madres 32.0\%, padres 27.3\%), con edad promedio
materna de 27.85 años y paterna de 30.42 años (Cuadro~\ref{tab:edad_padres}). El
nivel educativo predominante fue diversificado tanto en madres (40.90\%) como en
padres (48.50\%), observándose mayor proporción de educación superior en padres 
(Cuadro~\ref{tab:escolaridad}).

Respecto al empleo, el 50.0\% de madres no trabajaban, mientras que el 63.01\% 
de padres tenían empleo formal (Cuadro~\ref{tab:empleo}). Estas diferencias 
significativas en patrones de empleo por género ($\chi^2 = 1089.2$, 
$p < 0.001$) reflejan roles tradicionales de género en la región.

El acceso a servicios básicos fue favorable: 90.55\% con agua potable, 93.62\% 
con saneamiento adecuado, 100\% con electricidad y 87.13\% con gas propano 
para cocinar (Cuadro~\ref{tab:servicios_basicos}). La composición familiar
mostró predominio de hogares de 3-4 personas (43.54\%), con promedio de 1.21 
hermanos y 40.23\% de primogénitos (Cuadro~\ref{tab:composicion_familiar}).

La cobertura prenatal fue adecuada con promedio de 6.70 controles prenatales, 
94.55\% con acceso a ultrasonido obstétrico, 88.17\% consumieron vitaminas
prenatales en el primer trimestre del embarazo y 92.93\% consumieron vitaminas
prenatales durante el resto del embarazo (Cuadro~\ref{tab:cuidados_prenatales}). 
La mayoría de los partos fueron atendidos en hospitales públicos (64.7\%) con 
resolución por vía vaginal (59.0\%) (Figura~\ref{fig:parto}). Se encontraron
mejores condiciones en la región comparado con resultados de la Encuesta
Nacional de Salud Materno Infantil de 2014-2025 que reportaban menor
suplementación con ácido fólico y menor calidad de atención antenatal
\cite{Santos2025}.

Las principales causas de cesárea de emergencia fueron complicaciones del 
líquido amniótico (43.4\%), sufrimiento fetal (26.4\%) y preeclampsia (11.3\%) 
(Cuadro~\ref{tab:causas_cesarea}).

La lactancia materna exclusiva mostró tendencia decreciente con la edad: 72.7\% 
(0-6 meses), 55.6\% (6-12 meses) y 48.1\% (12-24 meses), con incremento 
correspondiente de lactancia mixta (Cuadro~\ref{tab:lactancia_periodos}). La 
suplementación con vitamina A disminuyó de 69.7\% (6-12 meses) a 53.9\% 
(18-24 meses) (Cuadro~\ref{tab:vitamina_a}), patrón similar para vitaminas y 
minerales espolvoreados (Cuadro~\ref{tab:vitaminas_minerales}). Esta disminución
podría relacionarse con menor asistencia a servicios de salud a mayor edad del
niño.

Los antecedentes nutricionales mostraron retardo de crecimiento en 11.1\% y 
desnutrición aguda en 5.9\% (Cuadro~\ref{tab:antecedentes_nutricionales}). Las
hospitalizaciones fueron 12.0\% neonatales y 13.7\% en infancia 
(Cuadro~\ref{tab:hospitalizacion}), con ictericia neonatal (33.8\%) y neumonía 
(50.2\%) como principales causas respectivas
(Cuadro~\ref{tab:causas_hospitalizacion}). La cobertura de vacunas alcanzó
94.9\% (Cuadro~\ref{tab:vacunacion}).

Los resultados del Cuestionario Edades y Etapas 3 revelaron desarrollo adecuado
en comunicación (94.4\%),  resolución de problemas (92.6\%), desarrollo
socio-individual (92.8\%),  motricidad fina (91.4\%) y motricidad gruesa
(83.3\%)  (Cuadro~\ref{tab:desarrollo_dominios}). La motricidad gruesa mostró
mayor  vulnerabilidad con 16.7\% en riesgo, confirmando los hallazgos del
análisis de puntajes Z donde este dominio presentó la media más baja (-0.269) 
(Cuadro~\ref{tab:estadisticas_z_scores}). Se encontraron menores porcentajes de riesgo en los diferentes dominios del desarrollo comparado con estudios anteriores en Guatemala que reportaban hasta 40\% de desarrollo en riesgo en el dominio de motricidad fina \cite{Angulo2023}.

La clasificación global del desarrollo mostró que 66.96\% de niños tienen 
desarrollo adecuado global, 33.04\% presentan riesgo en cualquier dominio y 
5.22\% alto riesgo (Cuadro~\ref{tab:desarrollo_global}). Se encontraron menores porcentajes de riesgo en cualquier dominio del neurodesarrollo comparado con estudios de Argentina (45\%) y Perú (38.7\%) pero mayores con relación a poblaciones de Cuba (21.5\%) y México (14.7\%)
\cite{GuadarramaCelaya2011,Kyerematen2014,CarlosOliva2020,RicardoGarcell2022}.

\section{Análisis de asociación de variables}
El análisis estratificado por edad reveló patrones distintivos de riesgo en el 
neurodesarrollo con alta significancia estadística ($p<0.001$). Los 8 meses 
emergieron como el período de mayor vulnerabilidad (OR = 3.78, IC 95\%: 
[2.29, 6.24]), con una prevalencia de riesgo global del 63.8\%. Esta edad 
concentra múltiples asociaciones significativas con los dominios del desarrollo
(Figura~\ref{fig:prevalencia_riesgo_asq3}).

En contraste, se identificaron edades con patrones protectores notables: 6 meses 
(OR = 0.317, IC 95\%: [0.18, 0.56]), 12 meses (OR = 0.396, IC 95\%: [0.25, 0.62]) 
y 27 meses (OR = 0.272, IC 95\%: [0.14, 0.52]). Estos períodos mostraron la 
menor prevalencia de riesgo entre todos los grupos evaluados
(Figura~\ref{fig:odds_ratio_asq3}).

Se identificaron períodos adicionales de riesgo elevado a los 4, 16, 18, 20, 24 
y 60 meses, donde los odds ratios superaron el valor de 2.0, con prevalencias 
de riesgo superiores al 45\%. Esta distribución temporal de riesgos sugiere 
períodos críticos que requieren monitoreo específico del neurodesarrollo.
Estos patrones de edad y riesgo en el neurodesarrollo se acercan a las
recomendaciones de la Academia Americana de Pediatría en su programa Bright
Futures que propone de tamizar a los 9, 18, y 30 meses de edad, así como monitorear el desarrollo a los 2, 4, 6, 12, 15, 24, 36, 48, 60 meses \cite{Lipkin2020}. 

El análisis de asociación reveló que el sexo del niño no mostró diferencias 
significativas en ningún dominio del desarrollo: comunicación 
($\chi^2 = 2.986$, $p = 0.225$), motricidad gruesa ($\chi^2 = 1.376$, 
$p = 0.503$), motricidad fina ($\chi^2 = 1.591$, $p = 0.451$), resolución 
de problemas ($\chi^2 = 1.957$, $p = 0.376$) y socio-individual 
($\chi^2 = 0.340$, $p = 0.844$) (Cuadro~\ref{tab:sexo_nino_desarrollo_chi2}), 
contrastando con estudios que exponen diferencias significativas en las que
el sexo masculino se asocia con mayor riesgo de trastornos del neurodesarrollo \cite{Sudry2024,Christensen2025,Peyre2019,Nishimura2016}.

Sin embargo, el grupo étnico mostró asociaciones significativas en resolución 
de problemas ($\chi^2 = 4.57$, $p = 0.033$, OR = 1.50, IC 95\%: 1.05-2.16) 
y desarrollo socio-individual ($\chi^2 = 4.90$, $p = 0.027$, OR = 1.53, 
IC 95\%: 1.06-2.21), con mayor riesgo en población indígena. Los dominios sin 
asociación significativa fueron: comunicación ($\chi^2 = 2.73$, $p = 0.098$), 
motricidad gruesa ($\chi^2 = 0.39$, $p = 0.533$) y motricidad fina 
($\chi^2 = 1.89$, $p = 0.169$) 
(Cuadro~\ref{tab:grupo_etnico_desarrollo_chi2_compacta}). Este hallazgo 
sugiere que las disparidades étnicas reflejan inequidades en oportunidades de
desarrollo, la misma observación se ha encontrado en comunidades de Australia,
Canadá, Nueva Zelanda, Estados Unidos, Brasil y Argentina
\cite{Lau2022,Hanly2020,Wehby2017}.

El área de residencia mostró asociaciones robustas con riesgo en cuatro
dominios: comunicación ($\chi^2 = 8.70$, $p = 0.003$, OR = 2.09, 
IC 95\%: 1.30-3.36), motricidad gruesa ($\chi^2 = 12.62$, $p < 0.001$, 
OR = 1.79, IC 95\%: 1.31-2.46), resolución de problemas ($\chi^2 = 25.13$, 
$p < 0.001$, OR = 2.77, IC 95\%: 1.85-4.15) y socio-individual 
($\chi^2 = 15.06$, $p < 0.001$, OR = 2.31, IC 95\%: 1.52-3.51), todos con 
mayor riesgo en área rural. Solo motricidad fina no mostró asociación 
significativa ($\chi^2 = 1.73$, $p = 0.188$) 
(Cuadro~\ref{tab:area_residencia_desarrollo_chi2_compacta}). Esta asociación
refleja desventajas en la población rural de Quetzaltenango, lo que es
consistente con estudios de Estados Unidos, India, China y Pakistán
\cite{Zablotsky2020-tb,Chatterjee2020,Murthy2020,Wang2020,Avan2010}.

El nivel educativo materno se asoció significativamente con todos los dominios 
del desarrollo: comunicación ($F = 5.96$, $p < 0.001$), motricidad gruesa 
($F = 6.29$, $p < 0.001$), motricidad fina ($F = 8.68$, $p < 0.001$), 
resolución de problemas ($F = 9.51$, $p < 0.001$) y socio-individual 
($F = 6.44$, $p < 0.001$), donde el grupo universitario mostró puntajes más 
altos y diferencias significativas con otros grupos, mientras que ``ninguna 
escolaridad'' presentó valores más bajos 
(Cuadro~\ref{tab:nivel_educativo_madre_desarrollo_anova}). Esto es consistente
con resultados de estudios longitudinales realizados en Francia, Singapur y
Brasil que reportan menores puntajes en niños de madres con bajo nivel educativo
\cite{Charkaluk2024,Lockhart2023,Yeleswarapu2025,Munhoz2022}, al igual que en
estudios de Perú e Indonesia se hallaron mayores puntajes en todos los dominios
en madres con mayor nivel educativo \cite{Handal2007,Hanifah2022}.

El nivel educativo paterno se asoció con los dominios de comunicación
($F = 6.82$, $p < 0.001$) y socio-individual 
($F = 2.57$, $p = 0.036$), pero no con motricidad gruesa ($F = 1.88$, 
$p = 0.114$), motricidad fina ($F = 1.09$, $p = 0.359$) ni resolución de 
problemas ($F = 1.36$, $p = 0.247$) 
(Cuadro~\ref{tab:nivel_educativo_padre_desarrollo}). Esto se evidencia en países
como Dinamarca, Noruega, y Estados Unidos
\cite{Holstein2021,Torvik2020,Sauver2004}.

La situación laboral paterna se asoció específicamente con motricidad gruesa 
($\chi^2 = 8.00$, $p = 0.005$, OR = 1.78, IC 95\%: 1.21-2.62), donde niños 
con padres desempleados presentaron mayor riesgo. Los otros dominios no 
mostraron asociación significativa: comunicación ($\chi^2 = 0.57$, $p = 0.451$), 
motricidad fina ($\chi^2 = 0.51$, $p = 0.473$), resolución de problemas 
($\chi^2 = 0.09$, $p = 0.767$) y socio-individual ($\chi^2 = 1.80$, 
$p = 0.180$) (Cuadro~\ref{tab:situacion_laboral_padre_resumen_compacta}). 
Este hallazgo puede reflejar el impacto de inestabilidad económica en 
acceso a recursos nutricionales que favorecen el desarrollo motor.

La situación laboral materna no se asoció significativamente con riesgos en los
dominios del desarrollo.

El estado civil del cuidador mostró asociaciones significativas en los dominios
de comunicación ($F = 10.18$, $p < 0.001$), motricidad gruesa
($F = 5.03$, $p = 0.007$), 
resolución de problemas ($F = 6.12$, $p = 0.002$) y socio-individual 
($F = 3.88$, $p = 0.021$), donde niños de madres solteras presentaron 
menor desempeño. Solo motricidad fina no mostró asociación ($F = 1.60$, 
$p = 0.202$) (Cuadro~\ref{tab:estado_civil_cuidador_desarrollo}). El número 
de hermanos se asoció con comunicación ($F = 5.71$, $p < 0.001$), motricidad 
fina ($F = 3.70$, $p = 0.011$) y socio-individual ($F = 2.91$, $p = 0.034$), 
pero no con motricidad gruesa ($F = 2.44$, $p = 0.063$) ni resolución de 
problemas ($F = 0.78$, $p = 0.504$), donde familias con más de 3 hermanos 
mostraron mayor riesgo (Cuadro~\ref{tab:total_hermanos_desarrollo}).
Una revisión sistemática de la literatura encontró que los niños en núcleos
familiares de padres solteros tienen mayor riesgo de presentar TDAH
\cite{Claussen2022}.

El retardo de crecimiento se asoció con riesgo en motricidad gruesa
($\chi^2 = 4.60$, $p = 0.032$, OR = 0.66, IC 95\%: 0.46-0.95) y el dominio
socio-individual ($\chi^2 = 8.01$, $p = 0.005$, OR = 0.49, IC 95\%: 0.31-0.79),
pero no con dominios de comunicación ($\chi^2 = 0.81$, $p = 0.369$), motricidad
fina ($\chi^2 = 3.69$, $p = 0.055$) ni resolución de problemas ($\chi^2 = 0.43$, $p = 0.510$) 
(Cuadro~\ref{tab:retardo_crecimiento_chi2_compacta}). La desnutrición aguda 
se relacionó específicamente con riesgo en el desarrollo socio-individual
($\chi^2 = 4.05$, $p = 0.044$, OR = 1.97, IC 95\%: 1.07-3.64), pero no con otros
dominios:  comunicación ($\chi^2 = 1.50$, $p = 0.221$), motricidad gruesa 
($\chi^2 = 0.90$, $p = 0.342$), motricidad fina ($\chi^2 = 1.00$, $p = 0.316$) 
ni resolución de problemas ($\chi^2 = 1.30$, $p = 0.254$) 
(Cuadro~\ref{tab:desnutricion_aguda_resumen_compacta}), consistente con 
evidencia que sugiere que desnutrición tanto crónica como aguda afecta
competencias del desarrollo \cite{Babikako2022,Suryawan2021,vandenHeuvel2019}.

Las horas de exposición a pantallas (televisor, tablet, móvil) se asociaron
significativamente con riesgo en todos los dominios del desarrollo: comunicación
($F = 3.79$, $p = 0.010$), motricidad gruesa ($F = 2.62$, $p = 0.049$),
motricidad fina ($F = 2.99$, $p = 0.030$), resolución de problemas ($F = 3.75$,
$p = 0.011$) y  socio-individual ($F = 4.33$, $p = 0.005$), observándose mayor
riesgo con mayor exposición
(Cuadro~\ref{tab:horas_exposicion_pantallas_desarrollo}). La literatura propone
un modelo de disrupción de neurotransmisores como la acetilcolina, glutamato,
serotonina como el mecanismo de aumento de riesgo de trastornos del
neurodesarrollo, el tiempo en pantalla se asocia a déficit de atención,
trastornos del habla y lenguaje, y si la exposición es temprana puede asociarse
con trastorno del espectro autista, aunque se propone que actividades al aire
libre reducen este riesgo \cite{Priyadarshini2025,Zehra2025,Hill2024,Sarfraz2023,Amorim2023,Sugiyama2023,Goswami2023,Jourdren2023}.

Las horas de juego que el cuidador le dedica al niño se asociaron con riesgos en
dominios de motricidad gruesa ($F = 3.05$, $p = 0.028$) y resolución de
problemas ($F = 6.49$, $p < 0.001$), pero no con comunicación ($F = 1.19$,
$p = 0.314$), motricidad fina ($F = 2.57$, $p = 0.053$) ni socio-individual
($F = 0.23$, $p = 0.878$), donde el grupo con menos horas (0-1) mostró mayor
riesgo (Cuadro~\ref{tab:horas_juego_cuidador_desarrollo}). Las interacciones
entre cuidador y niño se asocian con mejores desenlaces del desarrollo, y son
factores protectores de desarrollo cognitivo y conductual en infantes de alto
riesgo de presentar trastornos del neurodesarrollo
\cite{Jaffee2007,Isaev2023,Schneider2022}.

\subsection{Factores sin asociación significativa}

Diversos factores no mostraron asociación significativa con el neurodesarrollo: 
edad materna y paterna ($p > 0.05$ en todos los dominios), total de personas en
el hogar ($p > 0.05$ en todos los dominios), número de controles prenatales
($p > 0.05$ en todos los dominios), vitaminas prenatales 
($p > 0.05$ en todos los dominios), 
tipo de parto ($p > 0.05$ en todos los dominios) 
(Cuadro~\ref{tab:tipo_parto_desarrollo}), servicio de asistencia al parto 
($p > 0.05$ en todos los dominios) 
(Cuadro~\ref{tab:servicio_asistencia_parto_desarrollo}), y antecedente de
hospitalización neonatal e infantil ($p > 0.05$ en todos los dominios) 
(Cuadros~\ref{tab:hospitalizado_neonatal_resumen} y 
\ref{tab:hospitalizado_infancia_resumen}). El acceso a seguridad social tampoco
mostró asociación significativa con riesgo en el neurodesarrollo ($p > 0.05$ en
todos los dominios). 

Este estudio presenta limitaciones inherentes al diseño transversal que impide 
establecer relaciones causales. La muestra, aunque representativa de la
población que acude a servicios de atención primaria, puede no reflejar
completamente características de la población general. La posible presencia de
factores de confusión no medidos, como calidad de interacción entre el cuidador
y el niño o condiciones médicas no diagnosticadas, o antecedentes médicos
erróneos podría influir en resultados observados.
