\section{Recomendaciones}

\subsection*{A las autoridades de salud}
\begin{enumerate}
\item Incorporar programas de tamizaje del neurodesarrollo infantil en 
servicios de primer y segundo nivel de atención.

\item Modificar el carné del niño y de la niña para incluir una sección 
específica de seguimiento de resultados del tamizaje del neurodesarrollo.

\item Implementar herramientas validadas como los Cuestionarios Edades y 
Etapas 3 (ASQ-3) o las Escalas Globales para el Desarrollo Temprano de la 
Organización Mundial de la Salud.

\item Priorizar el tamizaje intensivo a los 8 meses e incorporar el 
monitoreo en edades que coincidan con los períodos de vacunación (2, 4, 6, 
12, 18 y 48 meses).

\item Crear sistemas de referencia y contrarreferencia entre niveles de 
atención para casos con riesgo identificado en el tamizaje.

\item Destinar presupuesto específico para capacitación continua del 
personal de salud en aplicación e interpretación de herramientas de 
tamizaje.
\end{enumerate}

\subsection*{Al personal de salud}
\begin{enumerate}
\item Promover la importancia del tamizaje del neurodesarrollo infantil en 
la consulta de niño sano, especialmente a los 8 meses de edad.

\item Informar a los padres sobre actividades específicas de estimulación 
temprana e interacciones positivas cuidador-niño.

\item Elaborar charlas educativas sobre la importancia de los primeros mil 
días para el desarrollo infantil.

\item Utilizar material promocional de UNICEF y OMS para empoderar a los 
padres en el acceso a información sobre estimulación temprana.

\item Enfatizar en las consultas los riesgos de la exposición temprana y 
prolongada a dispositivos electrónicos en menores de 5 años.
\end{enumerate}

\subsection*{A los padres de familia}
\begin{enumerate}
\item Realizar actividades diarias de estimulación temprana estructuradas de acuerdo a la edad del niño:
\begin{itemize}
    \item 10-15 minutos de ejercicios de motricidad gruesa
    \item 10 minutos de estimulación del lenguaje con interacción responsiva
    \item 10 minutos de actividades de motricidad fina
\end{itemize}

\item Evitar la exposición temprana y prolongada de niños menores de 
5 años a pantallas de dispositivos electrónicos (televisores, tablets, 
teléfonos móviles), siguiendo las recomendaciones de la Academia Americana 
de Pediatría.

\item Promover el juego activo e interacciones directas entre 
cuidadores y niños como alternativa al uso de medios electrónicos, 
priorizando actividades que estimulen el desarrollo integral.

\item Permitir espacios de juego con otros niños para fortalecer 
habilidades socio-individuales y de comunicación.
\end{enumerate}

\subsection*{A las autoridades educativas}
\begin{enumerate}
\item Crear centros de desarrollo infantil temprano en comunidades rurales 
e indígenas con mayor vulnerabilidad.

\item Capacitar a maestros de educación inicial en identificación temprana 
de retrasos del desarrollo y uso de herramientas de tamizaje.

\item Desarrollar material educativo culturalmente apropiado sobre 
estimulación temprana para la población.
\end{enumerate}

\subsection*{A las autoridades departamentales}
\begin{enumerate}
\item Crear programas de estimulación temprana en coordinación 
con el sistema de salud, priorizando comunidades rurales.

\item Establecer centros comunitarios de desarrollo infantil en áreas 
rurales con población indígena.

\item Implementar programas de seguridad alimentaria y nutricional 
dirigidos a familias con niños menores de 5 años para prevenir 
desnutrición.

\item Desarrollar campañas de sensibilización sobre la importancia del 
desarrollo infantil temprano y los riesgos del uso excesivo de pantallas 
en medios de comunicación locales.

\item Facilitar espacios públicos seguros para el juego infantil y la 
interacción familiar en comunidades urbanas y rurales.
\end{enumerate}

\section{Aportaciones del estudio}
\begin{enumerate}
\item Sensibilización de desigualdades sociales que afectan el neurodesarrollo. 
Este estudio evidencia cómo factores como el nivel educativo de los padres, la 
etnicidad y la zona de residencia constituyen barreras estructurales que 
limitan el pleno desarrollo de los niños. Los hallazgos proporcionan evidencia 
cuantitativa de disparidades previamente documentadas de manera fragmentada.

\item Generación de evidencia local para la toma de decisiones basada en 
evidencia. A diferencia de estudios generalizados, este trabajo aporta datos 
específicos del contexto de Quetzaltenango, lo cual permite adaptar las 
estrategias de intervención a las realidades socioculturales de la región y 
facilita el diseño de políticas públicas contextualizadas.

\item Énfasis en la importancia del entorno familiar y social en el desarrollo 
infantil. Más allá de los factores médicos tradicionales, el estudio demuestra 
que el entorno sociocultural en el que crece el niño tiene un peso determinante 
en su desarrollo. Esta evidencia refuerza la necesidad de implementar 
intervenciones integrales que aborden tanto al niño como a su entorno familiar 
y comunitario.

\item Fundamento para futuras investigaciones y líneas de investigación. Los 
resultados de esta investigación establecen una base sólida para profundizar 
en estudios cualitativos sobre prácticas de crianza, percepción del desarrollo 
infantil en comunidades indígenas, y análisis longitudinales sobre el impacto 
de intervenciones educativas dirigidas a madres con bajo nivel escolar. Estos 
hallazgos demuestran el potencial impacto de intervenciones dirigidas sobre el 
desarrollo infantil.
\end{enumerate}
