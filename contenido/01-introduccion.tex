\chapter{Introducción}
El neurodesarrollo infantil constituye un proceso dinámico y complejo que 
establece las bases fundamentales del futuro cognitivo, emocional y social de 
los individuos. Durante los primeros años de vida, especialmente en los primeros
1,000 días, las experiencias tempranas y las interacciones con el entorno
moldean la arquitectura cerebral, determinando habilidades cruciales como el
lenguaje, la memoria, la motricidad y el control emocional. Los trastornos en
este ámbito trascienden el impacto individual, generando una carga significativa
sobre las familias, los sistemas educativos y de salud pública.

A nivel mundial, UNICEF reporta que aproximadamente 250 millones de niños
menores de 5 años están en riesgo de no alcanzar su potencial de desarrollo,
mientras  que cerca de 200 millones presentan retrasos en su desarrollo global
debido a  la desnutrición en la primera infancia \cite{UNICEF2023}. Esta
realidad se  intensifica particularmente en países de ingresos bajos y medios,
donde las  adversidades socioeconómicas, la pobreza y el acceso limitado a
servicios de  salud crean un entorno adverso para el desarrollo infantil óptimo.

Guatemala presenta uno de los panoramas más desafiantes de Latinoamérica en
materia de desarrollo infantil. Según el informe de la línea de base de la Gran
Cruzada Nacional por la Nutrición 2021/2022, únicamente el 49.8\% de los niños
guatemaltecos entre 24 y 59 meses se encuentran en el camino adecuado de
desarrollo, salud, aprendizaje y bienestar psicosocial. Esta situación se agrava
dramáticamente por el limitado acceso a programas de primera infancia,
evidenciado por el hecho de que solo el 1.9\% de las madres de niños entre 2 y 5
años reportaron que sus hijos habían participado en estos programas
\cite{SESAN2022}.

El contexto guatemalteco presenta desafíos extraordinarios que se manifiestan en
profundas desigualdades estructurales. La desnutrición crónica afecta al 46.5\%
de los niños menores de 5 años, con disparidades significativas entre población
indígena (58\%) y no indígena (34.2\%), así como entre áreas rurales (53\%) y
urbanas (34.6\%) \cite{EnMaternoInfantil}. Estas desigualdades se reflejan
directamente en los indicadores de desarrollo: mientras que el 57.9\% de los
niños no indígenas muestra un desarrollo adecuado, solo el 45\% de los niños
indígenas alcanza este nivel, evidenciando las profundas brechas que
caracterizan al país \cite{SESAN2022}.

En la región de Quetzaltenango, estos desafíos se intensifican debido a la
particular convergencia de factores sociodemográficos, económicos y ambientales
que caracterizan esta zona. A pesar de constituir una de las regiones
económicamente más activas del país, persisten importantes desafíos en términos
de equidad en el acceso a oportunidades de desarrollo infantil.

Esta investigación surgió de la necesidad de generar evidencia científica para
comprender los factores de riesgo asociados al neurodesarrollo infantil en
uetzaltenango. El objetivo principal fue establecer la asociación entre
factores sociodemográficos, económicos, familiares y médicos con el riesgo en el
neurodesarrollo en niños menores de 5 años en servicios de atención primaria en
el distrito de Quetzaltenango, mediante evaluaciones con el \asq.
Específicamente, se buscó clasificar los resultados según grupos de edad para
detectar patrones específicos de riesgo, evaluar la asociación entre diversos
factores de riesgo y el neurodesarrollo, y examinar la relación entre el acceso
a servicios de salud y la presencia de riesgo de retraso en el desarrollo.

Los hallazgos de este estudio revelan datos alarmantes de la realidad en la que
se desarrollan miles de niños en Quetzaltenango. Se identificó que 33.04\% de
los niños presentan riesgo en al menos un dominio del desarrollo, siendo las
categorías de motricidad las más afectadas en la población. Asimismo, se
determinó que el bajo nivel educativo materno, la exposición temprana y
prolongada a pantallas, así como las desigualdades sociales, están estrechamente
relacionadas con desenlaces desfavorables en el neurodesarrollo infantil.

Los resultados de este estudio constituyen evidencia fundamental para contribuir
al diseño de intervenciones y políticas públicas que puedan mejorar el
desarrollo infantil temprano en Quetzaltenango. Esta investigación trasciende el
ámbito académico y representa un llamado a la acción para mejorar el futuro de
a niñez guatemalteca. Cada niño que no alcanza su potencial de desarrollo
representa una oportunidad perdida, no solo para él y su familia, sino para el
progreso colectivo de nuestra sociedad. Se espera que estos hallazgos
constituyan los primeros pasos para transformar la realidad del desarrollo
infantil en Guatemala.
