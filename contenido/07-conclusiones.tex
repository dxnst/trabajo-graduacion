\chapter{Conclusiones}

\begin{enumerate}
\item
En el análisis de patrones de riesgo por edad se encontró que los 8 meses
constituyen el período con mayor prevalencia de riesgo del 63.8\% (OR = 3.78),
en contraste, las edades de 6 meses (OR = 0.317), 12 meses (OR = 0.396) y 27
meses (OR = 0.272) mostraron patrones protectores, indicando períodos de menor
riesgo en el desarrollo. Otras edades con alto riesgo incluyen 4, 16, 18, 20, 24
y 60 meses, todas con prevalencia superior al 45\%, evidenciando momentos
críticos adicionales que requieren vigilancia específica.

\item
Respecto a patrones de riesgo por dominios del neurodesarrollo,
la motricidad gruesa fue el área más afectada (16.7\%), seguida por motricidad
fina (8.6\%), desarrollo socio-individual (7.2\%), resolución de problemas
(7.4\%) y comunicación (5.6\%), en el total de la población.

\item
Los factores sociodemográficos mostraron asociaciones significativas con el
riesgo en el neurodesarrollo. La etnia indígena presentó 1.50 veces mayor
riesgo en resolución de problemas y 1.53 en desarrollo socio-individual. La
residencia rural se asoció con riesgo elevado en comunicación (OR = 2.09),
motricidad gruesa (OR = 1.79), resolución de problemas (OR = 2.77) y
desarrollo socio-individual (OR = 2.31), evidenciando las disparidades
territoriales en oportunidades de desarrollo.

En cuanto a factores familiares, el nivel educativo materno bajo fue el
factor más consistente, afectando todos los dominios del desarrollo
($p < 0.001$). El nivel educativo paterno bajo se asoció con comunicación
($p < 0.001$) y desarrollo socio-individual ($p = 0.036$). El desempleo
paterno mostró asociación específica con riesgo en motricidad gruesa
(OR = 1.78). Respecto a la estructura familiar, los primogénitos presentaron
mayor riesgo en motricidad fina.

Los factores ambientales también demostraron influencia significativa. El
tiempo de exposición prolongada a pantallas electrónicas mostró asociación
con riesgo en todos los dominios del desarrollo ($p < 0.05$), evidenciando
que a mayor tiempo de exposición, mayor riesgo de alteraciones en el
neurodesarrollo. En contraste, las horas de juego entre cuidadores y niños
demostraron ser un factor protector significativo, especialmente para
motricidad gruesa ($p = 0.028$) y resolución de problemas ($p < 0.001$),
confirmando la importancia de la interacción social directa para el
desarrollo óptimo.

\item La atención prenatal, incluyendo número de controles y suplementación,
no mostró asociaciones significativas con el riesgo en el neurodesarrollo
($p > 0.05$). Sin embargo, los factores nutricionales postnatales sí
evidenciaron asociaciones importantes. El retardo de crecimiento, presente en
11.1\% de la muestra, incremento el riesgo en motricidad gruesa (OR = 0.66)
y desarrollo socio-individual (OR = 0.49). La desnutrición aguda, con
prevalencia de 5.9\%, se asoció específicamente con riesgo en desarrollo
socio-individual (OR = 1.97). Otros factores médicos como edad gestacional,
tipo de parto y lugar de atención del parto no mostraron relación con el
riesgo en el neurodesarrollo, sugiriendo que los determinantes sociales
prevalecen sobre los factores biológicos en el contexto estudiado.

\item Este estudio evidenció que el riesgo en el neurodesarrollo infantil en
Quetzaltenango está determinado por múltiples factores interrelacionados,
destacando la edad crítica de 8 meses, el predominio de riesgo en motricidad
gruesa y la influencia significativa de determinantes sociales como etnia
indígena, residencia rural y nivel educativo parental principalmente el materno.
La exposición prolongada a pantallas representa un factor de riesgo evidente,
mientras que el tiempo de juego cuidador-niño actúa como factor protector.
Los hallazgos resaltan la necesidad de implementar estrategias preventivas
enfocadas en los períodos críticos identificados, priorizando poblaciones
vulnerables y promoviendo interacciones positivas entre cuidadores y niños
en lugar de la exposición a dispositivos electrónicos.
\end{enumerate}
