\chapter*{Dedicatoria}
A Dios: Gracias, gracias Dios, por sostenerme cuando mis fuerzas claudicaban, 
por abrir caminos donde solo veía muros y por recordarme que cada tropiezo 
también es aprendizaje. Este logro es testimonio de tu fidelidad incalculable 
en mi vida.

A la vida, gracias por las pruebas que formaron mi carácter y por las alegrías 
que me recordaron por qué vale la pena insistir. Hoy cierro una etapa y abro 
otra con la certeza de que seguiré creciendo, sirviendo y aprendiendo.

A mis padres, gracias por no soltarme la mano en los días largos ni en las 
noches de estudio. Este título también lleva sus nombres.

A mi mami, Angélica Vásquez, eres el modelo de la mujer que aspiro a ser: 
valiente, trabajadora, amorosa y firme. Tus abrazos me devolvieron la calma, 
tus palabras me devolvieron la fe y tu ejemplo me enseñó que la perseverancia 
transforma los sueños en metas y las metas en realidad. Quiero ser como tú.

A mi papi, Edgar Ixquiac, tus consejos han sido guía en mi camino. Gracias por 
creer en mí, por tu paciencia, por enseñarme a pensar con claridad y a decidir 
con corazón y cabeza. Tus palabras me acompañarán en cada turno y en cada 
consulta.

A mi abuelita, Valen ($\dagger$), querida, aunque ya no estés físicamente, tu 
amor sigue vivo en mí. Gracias por tu ternura, por cuidar de mí cuando más 
necesitaba atención y cariño, por las oraciones silenciosas, por enseñarme que 
la bondad también sana, y sobre todo por enseñarme que la fidelidad de Dios 
sostiene la vida, y que creer en Él es suficiente siempre. Este logro te honra.

A Mariana Guzmán, gracias por tu compañía incondicional en un camino que no ha 
sido fácil. Durante el trabajo hospitalario me regalaste tu tiempo, tu apoyo y 
tu cariño sin medida. En los días más pesados fuiste descanso, y en los días 
de logro, celebración. Gracias por estar.

A mis padrinos, Paola Cifuentes y Yubini Mérida, gracias por sus consejos 
oportunos y su apoyo especial en cada etapa. Sus palabras sabias, su guía y su 
fe en mi formación hicieron la diferencia cuando la ruta parecía incierta. Los 
llevo conmigo en este logro y en los que vendrán.

Agradezco profundamente a esta casa de estudios que me abrió sus puertas y me 
permitió encontrar valiosos mentores. En especial, a las doctoras Norma Yxquiac 
y Sucelly Maaz, quienes con su ejemplo me enseñaron la importancia del servicio. 
Su guía ha sido fundamental en mi formación profesional y personal.

\begin{flushright}
Etelvina Del Rosario Ixquiac Vásquez
\end{flushright}
