\chapter*{Dedicatoria}
A mi madre, cuyo amor incondicional, sacrificio y apoyo constante han sido el 
pilar fundamental de mi formación académica y personal. A mi padre, por su 
paciencia y por enseñarme el valor de la perseverancia. Sus consejos y apoyo 
han sido invaluables en cada etapa de este camino.

A mis queridas abuela y bisabuela, cuya memoria y ejemplo de vida continúan 
inspirándome y recordándome la importancia del servicio hacia los demás y el 
compromiso con la comunidad.

A mis queridos amigos Sarah, Etelvina, Mariana, Sonia, Angela, Carlos, Jaime, 
Israel, Estuardo, Óscar, Giselle y Gabriela, por estar siempre presentes.

A los doctores Sucelly Maaz, Ricardo Guzmán, Salvador Soto y Priscila González, 
quienes participaron en diferentes etapas de esta investigación. Muchas gracias 
por la motivación y el apoyo para la realización de este estudio en beneficio 
de la niñez guatemalteca.

A la Dra. Norma Yxquiac, por su ardua labor en reducir las desigualdades en 
salud y su dedicación incansable al servicio de las comunidades más vulnerables 
de Guatemala.

A Aaron Swartz y Alexandra Elbakyan, por su esfuerzo en hacer la ciencia 
accesible para todos. El conocimiento debe compartirse entre todas las personas 
que lo necesitan, no solo entre aquellas que pueden pagarlo.

A la Universidad de San Carlos de Guatemala y al Instituto Guatemalteco de 
Seguridad Social, por ser parte fundamental de mi formación como profesional. 
Eternas gracias a todos los doctores cuya labor altruista y fuerza de voluntad 
contribuyen incansablemente a mejorar la salud de la población guatemalteca.

Finalmente, esta investigación está dedicada con especial cariño a todos los 
niños de Guatemala. Que este trabajo contribuya a crear intervenciones tempranas 
que les permitan alcanzar su máximo potencial de desarrollo, sin importar su 
origen étnico, condición socioeconómica o lugar de residencia. Ellos son la 
razón por la cual esta investigación cobra verdadero significado.

\vspace{1cm}

\begin{center}
\textit{``On ne voit clairement qu'avec le cœur. \\
L'essentiel est invisible aux yeux.''} \\
\textit{--- Antoine de Saint-Exupéry}
\end{center}

\vspace{1cm}

\begin{flushright}
Josué Daniel Soto Consuegra
\end{flushright}
